%% abtex2-modelo-artigo.tex, v-1.9.6 laurocesar
%% Copyright 2012-2016 by abnTeX2 group at http://www.abntex.net.br/
%%
%% This work may be distributed and/or modified under the
%% conditions of the LaTeX Project Public License, either version 1.3
%% of this license or (at your option) any later version.
%% The latest version of this license is in
%%   http://www.latex-project.org/lppl.txt
%% and version 1.3 or later is part of all distributions of LaTeX
%% version 2005/12/01 or later.
%%
%% This work has the LPPL maintenance status `maintained'.
%%
%% The Current Maintainer of this work is the abnTeX2 team, led
%% by Lauro César Araujo. Further information are available on
%% http://www.abntex.net.br/
%%
%% This work consists of the files abntex2-modelo-artigo.tex and
%% abntex2-modelo-references.bib
%%

% ------------------------------------------------------------------------
% ------------------------------------------------------------------------
% abnTeX2: Modelo de Artigo Acadêmico em conformidade com
% ABNT NBR 6022:2003: Informação e documentação - Artigo em publicação
% periódica científica impressa - Apresentação
% ------------------------------------------------------------------------
% ------------------------------------------------------------------------

\documentclass[
	% -- opções da classe memoir --
	article,			% indica que é um artigo acadêmico
	11pt,				% tamanho da fonte
	oneside,			% para impressão apenas no recto. Oposto a twoside
	a4paper,			% tamanho do papel.
	% -- opções da classe abntex2 --
	%chapter=TITLE,		% títulos de capítulos convertidos em letras maiúsculas
	%section=TITLE,		% títulos de seções convertidos em letras maiúsculas
	%subsection=TITLE,	% títulos de subseções convertidos em letras maiúsculas
	%subsubsection=TITLE % títulos de subsubseções convertidos em letras maiúsculas
	% -- opções do pacote babel --
	english,			% idioma adicional para hifenização
	brazil,				% o último idioma é o principal do documento
	sumario=tradicional
	]{abntex2}


% ---
% PACOTES
% ---

% ---
% Pacotes fundamentais
% ---
\usepackage{lmodern}			% Usa a fonte Latin Modern
\usepackage[T1]{fontenc}		% Selecao de codigos de fonte.
\usepackage[utf8]{inputenc}		% Codificacao do documento (conversão automática dos acentos)
\usepackage{indentfirst}		% Indenta o primeiro parágrafo de cada seção.
\usepackage{nomencl} 			% Lista de simbolos
\usepackage{color}				% Controle das cores
\usepackage{graphicx}			% Inclusão de gráficos
\usepackage{microtype} 			% para melhorias de justificação
% ---
		
% ---
% Pacotes adicionais, usados apenas no âmbito do Modelo Canônico do abnteX2
% ---
\usepackage{lipsum}				% para geração de dummy text
% ---
		
% ---
% Pacotes de citações
% ---
\usepackage[brazilian,hyperpageref]{backref}	 % Paginas com as citações na bibl
\usepackage[alf]{abntex2cite}	% Citações padrão ABNT
% ---

% ---
% Configurações do pacote backref
% Usado sem a opção hyperpageref de backref
\renewcommand{\backrefpagesname}{Citado na(s) página(s):~}
% Texto padrão antes do número das páginas
\renewcommand{\backref}{}
% Define os textos da citação
\renewcommand*{\backrefalt}[4]{
	\ifcase #1 %
		Nenhuma citação no texto.%
	\or
		Citado na página #2.%
	\else
		Citado #1 vezes nas páginas #2.%
	\fi}%
% ---

% ---
% Informações de dados para CAPA e FOLHA DE ROSTO
% ---
\titulo{Análise Bibliométrica com Bibliometrix, Linguagem R e Shiny}
\autor{Lena Lúcia de Moraes\thanks{lenamoraes@gmail.com} e Jerônimo Avelar Filho\thanks{jeroavf@gmail.com }}
\local{Brasil}
%\autor{Equipe \abnTeX\thanks{\url{http://www.abntex.net.br/}} \and Lauro
%César
%Araujo\thanks{laurocesar@laurocesar.com}}
\data{2017, v-1.0}
% ---

% ---
% Configurações de aparência do PDF final

% alterando o aspecto da cor azul
\definecolor{blue}{RGB}{41,5,195}

% informações do PDF
\makeatletter
\hypersetup{
     	%pagebackref=true,
		pdftitle={\@title},
		pdfauthor={\@author},
    	pdfsubject={Cientometria},
	    pdfcreator={LaTeX with abnTeX2},
		pdfkeywords={abnt}{latex}{abntex}{abntex2}{atigo científico},
		colorlinks=true,       		% false: boxed links; true: colored links
    	linkcolor=blue,          	% color of internal links
    	citecolor=blue,        		% color of links to bibliography
    	filecolor=magenta,      		% color of file links
		urlcolor=blue,
		bookmarksdepth=4
}
\makeatother
% ---

% ---
% compila o indice
% ---
\makeindex
% ---

% ---
% Altera as margens padrões
% ---
\setlrmarginsandblock{3cm}{3cm}{*}
\setulmarginsandblock{3cm}{3cm}{*}
\checkandfixthelayout
% ---

% ---
% Espaçamentos entre linhas e parágrafos
% ---

% O tamanho do parágrafo é dado por:
\setlength{\parindent}{1.3cm}

% Controle do espaçamento entre um parágrafo e outro:
\setlength{\parskip}{0.2cm}  % tente também \onelineskip

% Espaçamento simples
\SingleSpacing

% ----
% Início do documento
% ----
\begin{document}

% Seleciona o idioma do documento (conforme pacotes do babel)
%\selectlanguage{english}
\selectlanguage{brazil}

% Retira espaço extra obsoleto entre as frases.
\frenchspacing

% ----------------------------------------------------------
% ELEMENTOS PRÉ-TEXTUAIS
% ----------------------------------------------------------

%---
%
% Se desejar escrever o artigo em duas colunas, descomente a linha abaixo
% e a linha com o texto ``FIM DE ARTIGO EM DUAS COLUNAS''.
% \twocolumn[    		% INICIO DE ARTIGO EM DUAS COLUNAS
%
%---
% página de titulo
\maketitle

% resumo em português
\begin{resumoumacoluna}
 Conforme a ABNT NBR 6022:2003, o resumo é elemento obrigatório, constituído de
 uma sequência de frases concisas e objetivas e não de uma simples enumeração
 de tópicos, não ultrapassando 250 palavras, seguido, logo abaixo, das palavras
 representativas do conteúdo do trabalho, isto é, palavras-chave e/ou
 descritores, conforme a NBR 6028. (\ldots) As palavras-chave devem figurar logo
 abaixo do resumo, antecedidas da expressão Palavras-chave:, separadas entre si por
 ponto e finalizadas também por ponto.

 \vspace{\onelineskip}

 \noindent
 \textbf{Palavras-chave}: Bibliometria. Colaboração Científica. Bibliometrix. Linguagem R. Shiny.
\end{resumoumacoluna}

% ]  				% FIM DE ARTIGO EM DUAS COLUNAS
% ---

% ----------------------------------------------------------
% ELEMENTOS TEXTUAIS
% ----------------------------------------------------------
\textual

% ----------------------------------------------------------
% Introdução
% ----------------------------------------------------------
\section*{Introdução}
\addcontentsline{toc}{section}{Introdução}
Responsável prof. Ricardo

% ----------------------------------------------------------
% Seção de explicações
% ----------------------------------------------------------
\section{Metodologia}


\subsection{Consultas nas Bases: \emph{Web Of Science (WOS), Scopus, PubMed e Scielo}}

Foram realizadas consultas\footnote{As consultas foram realizadas por meio do Portal de Periódicos da CAPES (http://www.periodicos.capes.gov.br), com acesso remoto via CAFe. Acesso em 19/08/2017.} com os termos \emph{scientometry} e \emph{cientometric} nas bases \emph{WOS, Scopus, PubMed} e \emph{Scielo} . As consultas foram realizadas sem restrições, ou seja, não foram realizados filtros de ano, linguagem, país, tipo de documento etc. As consultas retornaram um grande número de artigos de diversas categorias/áreas de conhecimento, com predominância nas áreas da Ciência da Informação (CI) (Tabela ~\ref{tab:resultadoPesquisa}).

Os metadados dos artigos foram salvos na máquina local para posterior tratamento e análise. Vale ressaltar que o formato dos metadados das bases são diferentes e assim, cada base foi analisada separadamente. A \emph{Scopus} retornou o maior número documentos, mas limitou o download dos documentos. Logo, neste trabalho a base da \emph{Scopus} foi limitada aos 2000 artigos de pesquisas mais recentes. Na \emph{WOS, PubMed} e \emph{Scielo} foi possível o download de todos os documentos retornados. Na base de dados da \emph{Scielo}, além dos termos em inglês, foram usados os termos correspondentes em português: cientometria e cientométrica.


\begin{table}[!tp]
\caption{Resultado das buscas nas bases WOS, Scopus, PubMed e Scielo.} \label{tab:resultadoPesquisa}
\centering
\begin{tabular}{|p{1.3cm}|p{3cm}|p{1.5cm}|p{7.5cm}|}
  \hline
  % after \\: \hline or \cline{col1-col2} \cline{col3-col4} ...
 \textbf{ Base} & \textbf{Termos de Consulta} &\textbf{No. Artigos} & \textbf{Principais Categorias/Áreas de Pesquisa} \\ \hline%\hline
  WOS   & TS=(scientometry OR scientometric) & 1284 & Information Science Library Science (625), Computer Science Interdisciplinary Applications (424), Computer Science Information Systems (156), Multidisciplinary Sciences (72).\\ \hline
   Scopus & scientometry  OR  scientometric  & 6714 (2000 download) & Social Sciences (3,021), Computer Science (3,015) , Medicine (927), Decision Sciences (629), Business, Management and Accounting (553). \\ \hline
   PubMed & (scientometry) OR scientometric  & 540 & Não disponibilizado.  \\ \hline
   Scielo & (cientometria) OR (cientométrica) OR (scientometry) OR (scientometric) & 138 &  Ciência da informação e biblioteconomia (30), Biologia (12), Saúde pública, ambiental e ocupacional (10), Ciências multidisciplinares (8).\\ \hline
\end{tabular}
\end{table}

\subsection{Tratamento dos Dados}

\subsection{Leitura dos Dados}

\subsection{Análise dos Dados}

\subsubsection{Análise Exploratória}

\subsubsection{Análise Bibliométrica}
\section{Ferramentas}
    Responsbilidade do Marcelo
\section{Resultados}

% ---
% Finaliza a parte no bookmark do PDF, para que se inicie o bookmark na raiz
% ---
\bookmarksetup{startatroot}%
% ---

% ---
% Conclusão
% ---
\section*{Considerações finais}
\addcontentsline{toc}{section}{Considerações Finais}

%\lipsum[1]

\begin{citacao}
%\lipsum[2]
\end{citacao}

%\lipsum[3]

% ----------------------------------------------------------
% ELEMENTOS PÓS-TEXTUAIS
% ----------------------------------------------------------
\postextual

% ---
% Título e resumo em língua estrangeira
% ---

% \twocolumn[    		% INICIO DE ARTIGO EM DUAS COLUNAS

% titulo em inglês
\titulo{Bibliometric Analysis with Bibliometrix, Language R and Shiny}
\emptythanks
\maketitle

% resumo em português
\renewcommand{\resumoname}{Abstract}
\begin{resumoumacoluna}
 \begin{otherlanguage*}{english}
   According to ABNT NBR 6022:2003, an abstract in foreign language is a back
   matter mandatory element.

   \vspace{\onelineskip}

   \noindent
   \textbf{Keywords}:  Bibliometry. Scientific Collaboration. Bibliometrix. Language R. Shiny.
 \end{otherlanguage*}
\end{resumoumacoluna}

% ]  				% FIM DE ARTIGO EM DUAS COLUNAS
% ---

% ----------------------------------------------------------
% Referências bibliográficas
% ----------------------------------------------------------
\bibliography{abntex2-modelo-references}

% ----------------------------------------------------------
% Glossário
% ----------------------------------------------------------
%
% Há diversas soluções prontas para glossário em LaTeX.
% Consulte o manual do abnTeX2 para obter sugestões.
%
%\glossary

% ----------------------------------------------------------
% Apêndices
% ----------------------------------------------------------

% ---
% Inicia os apêndices
% ---
\begin{apendicesenv}

% ----------------------------------------------------------
\chapter{Apêndice}
% ----------------------------------------------------------
%\lipsum[55-57]

\end{apendicesenv}
% ---

% ----------------------------------------------------------
% Anexos
% ----------------------------------------------------------
\cftinserthook{toc}{AAA}
% ---
% Inicia os anexos
% ---
%\anexos
\begin{anexosenv}

% ---
\chapter{Anexo}
% ---

%\lipsum[31]

\end{anexosenv}

\end{document}
